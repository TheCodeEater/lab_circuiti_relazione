\documentclass[12pt,a4paper]{article}

\title{Filtro crossover} %argomento
\date{17 e 24 aprile, 9 maggio 2024}
\author{Giacomo Errani 0001078021 Paolo Forni 0001089375}

\setlength{\parindent}{20pt}
\usepackage{setspace}
\usepackage[margin=2cm]{geometry}
\usepackage{float}
\usepackage{amsfonts}
\usepackage{pdfpages}
\usepackage{verbatim}
\usepackage[format=plain, justification=justified, font={small, it}, labelfont=bf]{caption}
\usepackage{subcaption}
\usepackage{textgreek}
\usepackage{wrapfig}

\usepackage{fontspec}

\fontspec[Path=Fonts/, UprightFont=*, BoldFont=*b, ItalicFont=*i]{calibri.ttf}


\usepackage{graphicx}
\graphicspath{{./images}} %cartella immagini, separata dalla cartella files

\usepackage{listings}
\usepackage{xcolor}
\usepackage{hyperref}
\usepackage{amsmath}

\usepackage{subfiles} % Best loaded last in the preamble

% Imposto la path giusta per i font e le immagini
\ifSubfilesClassLoaded{ % Se sto compilando un subfile
  \fontspec[Path=../Fonts/, UprightFont=*, BoldFont=*b, ItalicFont=*i]{calibri.ttf}
} { % Se sto compilando il main
  \fontspec[Path=Fonts/, UprightFont=*, BoldFont=*b, ItalicFont=*i]{calibri.ttf}
}

\definecolor{codegreen}{rgb}{0,0.6,0}
\definecolor{codegray}{rgb}{0.5,0.5,0.5}
\definecolor{codepurple}{rgb}{0.58,0,0.82}
\definecolor{backcolour}{rgb}{0.95,0.95,0.92}

\lstdefinestyle{code_style}{
  backgroundcolor=\color{backcolour},
  commentstyle=\color{codegreen},
  keywordstyle=\color{magenta},
  numberstyle=\tiny\color{codegray},
  stringstyle=\color{codepurple},
  basicstyle=\ttfamily\footnotesize,
  breakatwhitespace=false,
  breaklines=true,
  captionpos=b,
  keepspaces=true,
  numbers=left,
  numbersep=5pt,
  showspaces=false,
  showstringspaces=false,
  showtabs=false,
  tabsize=2
}

% Some definitions
\newcommand{\numberthis}{\addtocounter{equation}{1}\tag{\theequatio}}
% usage: https://tex.stackexchange.com/questions/42726/align-but-show-one-equation-number-at-the-end

% Come dice Giorgia: bisogna parlare in italiano!!
\renewcommand{\figurename}{Figura}
\renewcommand{\tablename}{Tabella}

\newcommand{\theoryF}{ $(1442 \pm 10)$ Hz}
\newcommand{\amplitudeF}{$(1621 \pm 1)$ Hz}
\newcommand{\phaseF}{$(1437 \pm 2)$ Hz}

\begin{document}

\maketitle

\begin{abstract}

  In questo esperimento abbiamo ricreato un filtro crossover a due canali e ne abbiamo analizzato il comportamento
  in risposta ad un'onda sinusoidale. \\
  Innanzitutto abbiamo calcolato il valore atteso della frequenza di cross \theoryF \hspace{1pt} mediante la misura
  diretta dell'induttanza sul ramo passa basso e della capacità sul ramo passa alto. \\
  Analizzando poi la variazione dell'ampiezza dei segnali sui due canali abbiamo ottenuto una frequenza di cross pari a
  \amplitudeF. Mentre analizzando lo sfasamento relativo, abbiamo ottenuto un valore di \phaseF. \\
  La frequenza ottenuta a partire dallo sfasamento è risultata compatibile con il valore teorico atteso, mentre quella
  ottenuta a partire dall'ampiezza no.

\end{abstract}


\section{Introduzione}\label{sec:introduzione}

  \subfile{Sections/1-Introduzione.tex}


\section{Apparato sperimentale}\label{sec:apparato-sperimentale}

  \subfile{Sections/2-Apparato_sperimentale.tex}


\section{Analisi}\label{sec:analisi}

  \subfile{Sections/3-Analisi.tex}

\section{Conclusioni}
In conclusione questa esperienza ha parzialmente confermato la validità del modello teorico ideale del filtro crossover. La stima della frequenza di cross mediante lo sfasamento è risultata compatibile con quella prevista così come il comportamento dello sfasamento relativo tra Tweeter e Woofer. 

L'analisi delle ampiezze ha prodotto risultati in disaccordo con la teoria, portando ad una stima della frequenza di cross non compatibile con quella attesa.

Tuttavia, il cross è avvenuto, per cui si può escludere un malfunzionamento dell'apparato. Riteniamo che siano intervenuti effetti reali non previsti dal nostro modello, oppure ulteriori errori nella misura.

 Infine, la compatibilità della frequenza di crossover stimata tramite $V_{in}$ con quella teorica rafforza la tesi che l'apparato sperimentale funzioni correttamente. Se ci fossero malfunzionamenti nell'apparato, ci aspetteremmo discrepanze anche nelle misurazioni di $V_{in}$. 
 
 Nel complesso ciò indica che la teoria utilizzata per prevedere le ampiezze potrebbe essere inadeguata o incompleta.

\section*{Appendice}
  \appendix

\section{Errori strumentali}\label{sec:errori_strumentali}

	\subfile{Sections/A-Errori_strumentali.tex}

\section{Formule e dimostrazioni}\label{sec:formule-e-dimostrazioni}

  \subfile{Sections/B-Formule_e_dimostrazioni.tex}


\section{Correzione dettagliata errori sistematici}\label{sec:correzione-dettagliata-errori-sistematici}

  \subfile{Sections/C-Correzione_dettagliata_errori_sistematici.tex}

\end{document}
