% Preamble
\documentclass[../Relazione_circuiti]{subfiles}

% Packages

\graphicspath{{\subfix{../images/}}}

% Document
\begin{document}

\subsection{Analisi preliminare qualitativa}

  \begin{figure}[H]
    \centering

    \begin{subfigure}[b]{0.3\textwidth}
      \centering
      \includegraphics[width=\textwidth]{Cross_waveform_500.jpeg}

      \caption{Segnali a 500Hz}
      \label{fig:signal_500}

    \end{subfigure}
    \hfill
    \begin{subfigure}[b]{0.3\textwidth}
      \centering
      \includegraphics[width=\textwidth]{Cross_waveform_1600.jpeg}

      \caption{Segnali a 1600Hz}
      \label{fig:signal_1600}

    \end{subfigure}
    \hfill
    \begin{subfigure}[b]{0.3\textwidth}
      \centering
      \includegraphics[width=\textwidth]{Cross_waveform_17000.jpeg}

      \caption{Segnali a 17kHz}
      \label{fig:signal_17k}

    \end{subfigure}
    \hfill

    \caption{Segnali osservati sui rami Woofer (blu) e Tweeter (rosso)
      a frequenza fissata. Le incertezze non sono visibili a causa della scala.}

    \label{fig:signal_waveforms}

  \end{figure}


  La Fig.\,\ref{fig:signal_waveforms} mostra la forma d'onda dei segnali osservati sui rami Woofer e Tweeter in risposta
  ad un segnale sinusoidale.
  La Fig.\,\ref{fig:signal_1600} mostra il comportamento nei pressi della frequenza di cross attesa, la Fig.
  \,\ref{fig:signal_500} a 1/3 e la Fig.\,\ref{fig:signal_17k} a circa 10 volte.

  Si osserva (Fig.\,\ref{fig:signal_1600}) che, coerentemente con quanto atteso, i segnali hanno ampiezza simile nei
  pressi di $f_{cross}$.
  A basse frequenze si osserva (Fig.\,\ref{fig:signal_500}) un'attenuazione del 60\% dell'ampiezza sul ramo Tweeter e
  nessuna alterazione sul Woofer, ad alte frequenze un'attenuazione sul Woofer dell'86\% e nessuna sul Tweeter
  (Fig.\,\ref{fig:signal_17k}). Questa prima analisi qualitativa è in accordo con la teoria.

\subsection{Analisi dell'ampiezza}

  \begin{figure}[H]
    \centering

    \begin{subfigure}[t]{=0.49\textwidth}

      \includegraphics[width=\textwidth]{cross_dataonly.png}

      \caption{Andamenti sperimentali di Tweeter, Woofer e Vin (segnale in ingresso nel filtro).}
      \label{fig: amplitude_dataonly}

    \end{subfigure}
    \hfill
    \begin{subfigure}[t]{=0.49\textwidth}

      \includegraphics[width=\textwidth]{cross_gain_amplitude.png}

      \caption
      {Andamenti sperimentali e fit dei guadagni in tensione.}
      \label{fig:cross_gain}
    \end{subfigure}

    \caption{
      Ampiezza misurata in funzione della frequenza (asse della frequenza in scala logaritmica).
      A causa della scala le incertezze sulle singole misure non sono visibili.
      L'andamento è rappresentato da una linea continua a causa dell'alta densità di punti.
    }

    \label{fig:cross_amplitude}

  \end{figure}

  La Fig.\,\ref{fig:cross_amplitude} mostra l'andamento di ampiezza dei segnali filtrati in funzione della frequenza.

  La relazione funzionale Ampiezza massima-Frequenza (misurata ai capi della resistenza di carico dei rami) dipende
  dalla tensione in ingresso sul filtro $V_{in}$.

  Tuttavia, a causa della presenza della resistenza interna nel generatore e come visibile in
  Fig.\,\ref{fig: amplitude_dataonly}, la $V_{in}$
  dipende dalla frequenza ed assumerla costante potrebbe introdurre errori.
  Pertanto abbiamo determinato i guadagni in tensione (Fig.\,\ref{fig:cross_gain}) dividendo le ampiezze osservate per
  il valore di $V_{in}$
  osservato alla relativa frequenza, propagando le incertezze in quadratura (poiché casuali).
  Gli andamenti seguiti da queste funzioni sono:

  \begin{equation*}
    G_{woofer} = \frac{R}{\sqrt{R^2+(\omega L)^2}} \qquad \quad %\label{eq:gain_woofer}
    G_{tweeter} = \frac{R}{\sqrt{R^2+(\frac{1}{\omega C})^2}} %\label{eq:gain_tweeter}
  \end{equation*}

%dove $V_{in}$ rappresenta la tensione in ingresso. %(assunta costante) pari a 0.65, osservata nel ramo Woofer nel
%    limite di basse frequenze e nel ramo Tweeter nel limite di alte frequenze.

%Le funzioni guadagno si ottengono semplicemente dividendo le eq date per vin (NOTA: da riscrivere, non voglio
%    duplicare le equazioni)

  Le incertezze sull'ampiezza sono state ottenute mediante fit sinusoidale alla forma d'onda osservata come descritto in
  sezione~\ref{sec:apparato-sperimentale}.
  Ad essa è stata aggiunta l'incertezza strumentale sulla misura delle tensioni secondo le specifiche di ELVIS, poiché
  quelle ottenute da fit sono risultate minori di quanto lo strumento è in grado di risolvere.%DA VERIFICARE BENE

  È stato effettuato un fit ai parametri $L$ e $C$ ($R$ fissata poiché non determina la frequenza di cross).
  Da essi è stata poi ricavata la frequenza di crossover secondo l'Eq.\,\eqref{eq:f_cross}.

  \begin{table}[H]
    \centering

    \begin{tabular}{c | c }

      %Heading
      Grandezza                      & Valore                 \\

      L                              & $(10.45 \pm 0.01)$ mH  \\
      C                              & $(921.46 \pm 0.05)$ nF \\
      $f_{cross}$                    & \amplitudeF            \\
      $\widetilde{\chi}^2_{woofer}$  & 5458.67                   \\
      $\widetilde{\chi}^2_{tweeter}$ & 2288.34

    \end{tabular}

    \caption{Risultati del fit alle funzioni guadagno in tensione. Le incertezze sono di tipo casuale.}
    \label{tab:fit_amplitude}

  \end{table}

  La Tab.\,\ref{tab:fit_amplitude} mostra i risultati del fit.
  Si osserva che la frequenza ottenuta non risulta compatibile con il valore teorico stimato \theoryF.
  Tuttavia, l'andamento della curva $V_{in}$ ci fornisce un metodo ulteriore per determinare la frequenza di cross.
  Infatti, si dimostra che a tale frequenza l'impedenza totale del filtro è massima.
  Usando la legge di Ohm simbolica, ne segue che la tensione ai capi del filtro deve essere massima.
  Cercando il massimo della curva sperimentale, otteniamo il valore $(1434 \pm 8)$~Hz, che risulta compatibile con il
  valore teorico stimato.
  L'incertezza è stata determinata considerando metà del passo di scansione utilizzato nei pressi di quella frequenza
  (come descritto in sezione 2, non è costante).
  Risulta pertanto essere un'incertezza massima.

  %discussione del non venuto
  \begin{figure}[H]
    \centering

    \begin{subfigure}[t]{=0.49\textwidth}
        \includegraphics[width=\textwidth]{cross_reduced_range.png}
        \caption{Confronto dati e fit.}
        \label{fig: amp_gain_fit_data_reduced}

    \end{subfigure}
    \hfill
    \begin{subfigure}[t]{=0.49\textwidth}
        \includegraphics[width=\textwidth]{cross_reduced_range_theoric.png}
    	\caption{Confronto dati, fit e previsione teorica. Si noti la traslazione del cross tra teoria e dati.}
    	\label{fig: amp_gain_fit_theoric_reduced}
    \end{subfigure}

    	\caption{Fit funzioni guadagno. Zoom nei pressi della frequenza di crossover. 		Incertezze non visibili a causa della scala.}
    	\label{fig: amp_gain_fit_reduced}


  \end{figure}

  La Fig.\,\ref{fig: amp_gain_fit_reduced} mostra gli andamenti del fit e dei valori sperimentali nei pressi della frequenza di crossover.
  Riguardo alla curva del Woofer, si può notare come questa si discosti oltre l'incertezza dai rispettivi dati a frequenze inferiori alla frequenza di cross, per poi diminuire la distanza da essi a frequenze superiori.

  Una situazione analoga, sebbene meno marcata, si osserva sulla curva del Tweeter, dove la curva del fit, entro le incertezze a frequenza inferiori al cross, si discosta progressivamente dai dati reali a frequenze superiori.
  Questa discrepanza è evidente nel valore del $\chi^2$, nell'ordine di $10^3$, molto maggiore al valore ottimale pari a 1. 
  Tuttavia, le incertezze ottenute sono mediamente 4 ordini di grandezza inferiori ai dati. E' plausibile che queste abbiano portato ad un valore eccessivamente elevato del $\chi^2$.
  
  Ciononostante, si osserva, seppur traslata, l'intersezione tra le due curve sperimentali. Il cross è avvenuto come da predizione teorica, ma risulta traslato di circa 183 Hz.
  
 

\subsection{Analisi della fase}

  \begin{figure}[H]
    \centering

    \begin{subfigure}{=0.49\textwidth}
      \centering
      \includegraphics[width=.99\textwidth]{phase_dataonly.png}
      \caption{Sfasamento Tweeter-Woofer (solo dati)}
      \label{fig:pdiff_dataonly}

    \end{subfigure}
    \hfill
    \begin{subfigure}{=0.49\textwidth}
      \centering
      \includegraphics[width=.99\textwidth]{phase_cross.png}
      \caption{Dati tweeter e sovrapposizione con fit}
      \label{fig:pdiff_fit_data}

    \end{subfigure}

    \caption{
      Fase del Tweeter misurata rispetto al Woofer (blu). Fase del Woofer misurata rispetto se stesso (giallo).
      La frequenza è in scala logaritmica. Le incertezze non sono visibili a causa della scala.
    }
    \label{fig:phase_diff}

  \end{figure}

  La Fig.\,\ref{fig:pdiff_dataonly} mostra lo sfasamento relativo Tweeter-Woofer (curva Tweeter in figura), che segue la

  relazione definita dall'Eq.\,\eqref{eq:p_diff}.
  Ai dati è stata assegnata un'incertezza casuale stimata mediante fit alla forma d'onda osservata per ogni frequenza,
  come descritto nella sezione~\ref{sec:apparato-sperimentale}.

  % Modificato
  Inoltre, poiché l'acquisizione dei due canali non è simultanea ma sequenziale (mediante multiplexer che itera sui
  canali Analog Input) si ha che tra di essi vi è uno sfasamento temporale tra l'inizio delle due acquisizioni
  $\tau_{aq}=1$ \textmu S, secondo le specifiche di ELVIS, che si riflette in un errore
  sistematico sulla fase, che risulta quindi essere inferiore al valore vero.
  % Modificato

  Questo è stato corretto aggiungendo alla fase misurata $ 2 \pi \nu \cdot \tau_{aq}$, dove $\nu$
  rappresenta la frequenza del segnale (si veda l'appendice per una spiegazione approfondita).

  In Fig.\,\ref{fig:pdiff_dataonly} si nota che la fase del Woofer rispetto se stesso, che ci aspettavamo essere 0
  poiché il trigger è sul medesimo canale, risulta in realtà non nulla ma oscillante attorno a questo valore e che
  l'ampiezza di dette oscillazioni aumenta all'aumentare della frequenza.
  Ciò è probabilmente dovuto al trigger della ELVIS, che non azionando al momento esatto l'acquisizione ha introdotto
  una fase sul canale del woofer, che come per il multiplexer dipende dalla frequenza dell'onda acquisita.
  Questo fenomeno è stato osservato anche sulla fase del tweeter e la cosa è coerente con il fatto che la fase del
  tweeter è misurata relativamente al Woofer, e quindi una fluttuazione attorno allo 0 della fase si ripercuote su tutte
  le altre misure.
  Abbiamo corretto queste fluttuazioni sul canale Tweeter sottraendo i valori misurati sul canale Woofer.
  Questo non ha influenzato il valore medio di sfasamento del Tweeter ad una data frequenza poiché la fase del Woofer è
  in media 0.

  Effettuando un fit con l'Eq.\,\eqref{eq:p_diff}, fissando $R$
  poiché la frequenza di cross non dipende da esso, abbiamo
  ricavato i valori $L$ e $C$, e grazie all'Eq.\,\eqref{eq:p_diff_cross} abbiamo stimato la frequenza di cross.


  \begin{table}[H]
    \centering

    \begin{tabular}{c | c }
%Heading
      Grandezza                & Valore                          \\
      \hline
      $L$                      & $ (11.82 \pm 0.03) $ mH         \\
      $C$                      & $ (1.037 \pm 0.002) $ \textmu F \\
      $f_{cross}$              & $ (1437 \pm 2) $ Hz             \\
      $\widetilde{\chi}^{\,2}$ & $49.62$

    \end{tabular}
    \caption{Risultati del fit dell'Eq.\,\eqref{eq:p_diff}}
    \label{tab:fit_phase}

  \end{table}
%ribadisco per chiarezza


%discussione fit
  Il valore ottenuto è risultato compatibile con il valore teorico atteso~\theoryF.
  La stima mediante analisi della fase risulta tuttavia più precisa, dato che presenta un'incertezza percentuale
  dello~0.14\% contro lo~0.69\% del valore teorico.
  Il $\chi^2$
  ottenuto risulta molto maggiore del valore ottimale pari a~1.
  In effetti, il fit si discosta in maniera significativa dai dati a basse frequenze, risultando tuttavia più
  accurato nei pressi della frequenza di cross e ad alte frequenze.
  Tuttavia questo non inficia la compatibilità del valore ottenuto con il valore atteso per la frequenza di crossover.
  Possiamo pertanto ritenere che il modello descriva adeguatamente il comportamento reale nei pressi della frequenza
  di cross e a frequenze superiori ad essa.


\end{document}