% Preamble
\documentclass[../Relazione_circuiti]{subfiles}

% Packages

\graphicspath{{\subfix{../images/}}}

% Document
\begin{document}

\subsection{Analisi preliminare qualitativa}

  \begin{figure}[H]
    \centering

    \begin{subfigure}[b]{0.3\textwidth}
      \centering
      \includegraphics[width=\textwidth]{Cross_waveform_500.jpeg}

      \caption{Segnali a 500Hz}
      \label{fig:signal_500}

    \end{subfigure}

    \hfill

    \begin{subfigure}[b]{0.3\textwidth}
      \centering
      \includegraphics[width=\textwidth]{Cross_waveform_1600.jpeg}

      \caption{Segnali a 1600Hz}
      \label{fig:signal_1600}

    \end{subfigure}

    \hfill

    \begin{subfigure}[b]{0.3\textwidth}
      \centering
      \includegraphics[width=\textwidth]{Cross_waveform_17000.jpeg}

      \caption{Segnali a 17kHz}
      \label{fig:signal_17k}

    \end{subfigure}
    \hfill


    \caption{Segnali osservati sui rami Woofer (blu) e Tweeter (rosso)
      a frequenza fissata. Le incertezze non sono visibili a causa della scala.}

    \label{fig:signal_waveforms}

  \end{figure}


  La Fig. \ref{fig:signal_waveforms} mostra la forma d'onda dei segnali osservati sui rami Woofer e Tweeter in risposta
  ad un segnale sinusoidale.
  La Fig. \ref{fig:signal_1600} mostra il comportamento nei pressi della frequenza di cross attesa, la Fig.
  \ref{fig:signal_500} a 1/3 e la Fig. \ref{fig:signal_17k} a circa 10 volte.

  Si osserva (Fig. \ref{fig:signal_1600}) che, coerentemente con quanto atteso, i segnali hanno ampiezza simile nei
  pressi di $f_{cross}$.
  A basse frequenze si osserva (Fig. \ref{fig:signal_500}) un'attenuazione del 60\% dell'ampiezza sul ramo Tweeter e
  nessuna alterazione sul Woofer, ad alte frequenze un'attenuazione sul Woofer dell'86\% e nessuna sul Tweeter (Fig.
  \ref{fig:signal_17k}). Questa prima analisi qualitativa è in accordo con la teoria.

\subsection{Analisi della frequenza misurata}

\subsection{Analisi dell'ampiezza}

  \begin{figure}[H]
    \centering

    \begin{subfigure}[t]{=0.49\textwidth}

      \includegraphics[width=\textwidth]{cross_dataonly.png}

      \caption{Andamenti sperimentali di Tweeter, Woofer e Vin (segnale in ingresso nel filtro,
        a valle della resistenza interna del generatore).}
      \label{fig: amplitude_dataonly}

    \end{subfigure}
    \hfill
    \begin{subfigure}[t]{=0.49\textwidth}

      \includegraphics[width=\textwidth]{cross_gain_amplitude.png}

      \caption
      {Andamenti sperimentali e fit dei guadagni in tensione.}
      \label{fig:cross_gain}
    \end{subfigure}

    \caption{Ampiezza misurata in funzione della frequenza (asse della frequenza in scala logaritmica). A causa della scala le incertezze sulle singole misure non sono visibili. L'andamento è rappresentato da una
      linea continua a causa dell'alta densità di punti.}
    \label{fig:cross_amplitude}

  \end{figure}

  La Fig. \ref{fig:cross_amplitude} mostra l'andamento di ampiezza dei segnali filtrati in funzione della frequenza.

  La relazione funzionale Ampiezza massima-Frequenza (misurata ai capi della resistenza di carico dei rami) dipende
  dalla tensione in ingresso sul filtro $V_{in}$.
  Tuttavia, a causa della presenza della resistenza interna nel generatore e come visibile in Fig.
  \ref{fig: amplitude_dataonly}, la $V_{in}$
  dipende dalla frequenza ed assumerla costante potrebbe introdurre errori.
  Pertanto abbiamo determinato i guadagni in tensione (Fig. \ref{fig:cross_gain}
  ) dividendo le ampiezze osservate per il valore di $V_{in}$
  osservato alla relativa frequenza, propagando le incertezze in quadratura. Gli andamenti seguiti da queste funzioni sono: 

  \begin{equation}
    \label{eq:gain_woofer}
    G_{woofer} = \frac{1}{\sqrt{R^2+(\omega L)^2}}
  \end{equation}

  \begin{equation}
    \label{eq:gain_tweeter}
    G_{tweeter} = \frac{1}{\sqrt{R^2+(\frac{1}{\omega C})^2}}
  \end{equation}

%dove $V_{in}$ rappresenta la tensione in ingresso. %(assunta costante) pari a 0.65, osservata nel ramo Woofer nel
%    limite di basse frequenze e nel ramo Tweeter nel limite di alte frequenze.

%Le funzioni guadagno si ottengono semplicemente dividendo le eq date per vin (NOTA: da riscrivere, non voglio
%    duplicare le equazioni)

  È stato effettuato un fit ai parametri $L$ e $C$ ($R$ fissata).
  Da essi è stata poi ricavata la frequenza di crossover secondo l'Eq. \eqref{eq:f_cross}.

  \begin{table}[H]
    \centering

    \begin{tabular}{c | c }

      %Heading
      Grandezza & Valore                 \\

      L         & $(10.13 \pm 0.01)$ mH  \\
      C         & $(953.39 \pm 0.05)$ nF \\
      $f_{cross}$ & $(1619 \pm 2)$ Hz \\
	$\chi^2_{woofer}$	& 15755.73  \\
	$\chi^2_{tweeter}$ & 19483.52  

    \end{tabular}

    \caption{Risultati del fit alle funzioni guadagno in tensione}
    \label{tab:fit_amplitude}

  \end{table}

  La Tab. \ref{tab:fit_amplitude} mostra i risultati del fit.
  Si osserva che la frequenza ottenuta non risulta compatibile con il valore teorico stimato \theoryF.
  Tuttavia, l'andamento della curva $V_{in}$ ci fornisce un metodo ulteriore per determinare la frequenza di cross.
  Infatti, si dimostra che a tale frequenza l'impedenza totale del filtro è massima.
  Usando la legge di Ohm simbolica, ne segue che la tensione ai capi del filtro deve essere massima.
  Cercando il massimo della curva sperimentale, otteniamo il valore $(1434 \pm 8) \;$ Hz.
  L'incertezza è stata determinata considerando metà del passo di scansione utilizzato nei pressi di quella frequenza
  (come descritto in sezione 2, non è costante). Risulta pertanto essere un'incertezza massima.
  



\subsection{Analisi della fase}

  \begin{figure}[H]
    \centering

    \begin{subfigure}{=0.49\textwidth}
      \centering
      \includegraphics[width=.99\textwidth]{phase_dataonly.png}
      \caption{Sfasamento Tweeter-Woofer (solo dati)}
      \label{fig: pdiff_dataonly}

    \end{subfigure}
    \hfill
    \begin{subfigure}{=0.49\textwidth}
      \centering
      \includegraphics[width=.99\textwidth]{phase_cross.png}
      \caption{Dati e sovrapposizione con fit}
      \label{fig: pdiff_fit_data}

    \end{subfigure}

    \caption{Fase del Tweeter misurata rispetto al Woofer (blu). Fase del Woofer misurata rispetto se stesso (giallo)
      . La frequenza è in scala logaritmica. Le incertezze non sono visibili a causa della scala.}
    \label{fig: phase_diff}

  \end{figure}

  La Fig.\,\ref{fig: pdiff_dataonly} mostra lo sfasamento relativo Tweeter-Woofer (curva Tweeter in figura), che segue la
  relazione definita dall'Eq.\,\eqref{eq:p_diff}.
  La Fig.\,\ref{fig: pdiff_dataonly} mostra lo sfasamento relativo Tweeter-Woofer (curva Tweeter in figura).
  Effettuando un fit con l'Eq.\,\eqref{eq:p_diff}, fissando $R$
      poiché la frequenza di cross non dipende da esso, abbiamo
      ricavato i valori $L$ e $C$, e grazie all'Eq.\,\eqref{eq:p_diff_cross} abbiamo stimato la frequenza di cross.


  \begin{table}[H]
    \centering

    \begin{tabular}{c | c | c}
%Heading
      Grandezza   & Valore              & Unità di misura \\
      \hline
      L           & $ 11.82 \pm 0.03 $  & mH              \\
      C           & $ 1.037 \pm 0.002 $ & $\mathrm{\mu}$F \\
      $f_{cross}$ & $ 1437 \pm 2$       & Hz

    \end{tabular}
    \caption{Risultati del fit dell'Eq.\,\eqref{eq:p_diff}}
    \label{tab:fit_phase}

  \end{table}
%ribadisco per chiarezza
  Ai dati è stata assegnata un'incertezza casuale stimata mediante fit alla forma d'onda osservata per ogni frequenza,
  come descritto in sezione 2.

  Inoltre, poiché l'acquisizione dei due canali non è simultanea ma sequenziale (mediante multiplexer che itera sui
  canali Analog Input) si ha che tra due canali acquisiti consecutivamente (come nel nostro caso) vi è uno sfasamento
  temporale tra l'inizio delle due acquisizioni, che risulta essere pari a $\tau_{aq}=1 \mu \mathrm{S}$
      secondo le specifiche di ELVIS, che si riflette in un errore sistematico sulla fase, che risulta essere inferiore
      al valore vero.

      Questo è stato corretto aggiungendo alla fase misurata  $ 2 \pi f \tau_{aq}$, dove $f$
      rappresenta la frequenza del segnale (si veda l'appendice per una spiegazione approfondita).

      In Fig.\,\ref{fig: phase_diff}
      si nota che la fase del Woofer rispetto se stesso, che ci aspettavamo essere 0 poiché il trigger è sul medesimo
      canale, risulta in realtà non nulla ma oscillante attorno a questo valore. L'ampiezza di dette oscillazioni
      aumenta all'aumentare della frequenza. Questo fenomeno è stato osservato anche sulla fase del tweeter e la cosa è
      coerente con il fatto che la fase del tweeter è misurata relativamente al Woofer: una fluttuazione nello 0 della
      fase si ripercuote su tutte le altre misure. Abbiamo corretto queste fluttuazioni sul canale Tweeter sottraendo i
      valori misurati sul canale Woofer. Questo non ha influenzato il valore medio di sfasamento del Tweeter ad una data
      frequenza poiché la fase del Woofer è in media 0.


\end{document}