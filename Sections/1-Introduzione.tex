% Preamble
\documentclass[../Relazione_circuiti]{subfiles}
% \usepackage{amstex} Mi si era aggiunto da solo, magari un giorno capiamo che serve

% Packages
\graphicspath{{\subfix{../images/}}}

% Document
\begin{document}

Il filtro crossover è un circuito RLC costituito da due rami: uno con un filtro passa basso realizzato mediante un
induttore e una resistenza in serie (ramo \textbf{Woofer}), l'altro con un filtro passa alto realizzato mediante
condensatore e resistenza in serie (ramo \textbf{Tweeter}). Le due resistenze sono identiche a meno dell'incertezza.

\begin{figure}[H]
  \centering
  \includegraphics[width=8cm]{Schema_crossover.png}

  \caption{Schema del circuito}
  \label{fig:schema_circuito}

\end{figure}

In risposta ad un segnale oscillante in ingresso, questo è attenuato in ampiezza dal ramo Woofer se a frequenza superiore ad una particolare frequenza (detta \textit{frequenza di crossover}),
attenuato dal ramo Tweeter se inferiore. 
Detta frequenza dipende dalla caratteristiche del circuito. Formalmente si definisce frequenza di crossover $f_{cross}$ la
frequenza a cui l'ampiezza dei segnali sui due canali è uguale e si dimostra che vale:

\begin{equation}
  \label{eq:f_cross}
  f_{cross} = \frac{1}{2 \pi \sqrt{LC} }
\end{equation}

dove $L$ è l'induttanza della bobina sul ramo Woofer e $C$ la capacità del condensatore sul ramo Tweeter.

Si ha inoltre, sempre alla suddetta frequenza, uno sfasamento dei due rami rispetto al generatore uguale e opposto: il
Woofer anticipa, il Tweeter ritarda.

\begin{align}
  \phi_{woofer} &= \arctan(-\frac{\omega L}{R}) \label{eq:p_woofer} \\
  \phi_{tweeter} &= \arctan(\frac{1}{\omega RC}) \label{eq:p_tweeter}
\end{align}

%\begin{equation} \label{eq:p_woofer}
%  \phi_{woofer} \; = \; \arctan(-\frac{\omega L}{R})
%\end{equation}
%\begin{equation} \label{eq:p_tweeter}
%  \phi_{tweeter} \; = \; \arctan(\frac{1}{\omega RC})
%\end{equation}

dove $L$ e $C$ sono definiti come nell'Eq. \eqref{eq:f_cross}, $\omega$ è la pulsazione dell'onda in ingresso, $R$ la resistenza
(identica su entrambi i rami).


\end{document}