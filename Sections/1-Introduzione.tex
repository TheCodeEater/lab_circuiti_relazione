% Preamble
\documentclass[../Relazione_circuiti]{subfiles}
% \usepackage{amstex} Mi si era aggiunto da solo, magari un giorno capiamo che serve

% Packages
\graphicspath{{\subfix{../images/}}}

% Document
\begin{document}

Questo esperimento vuole verificare la validità delle formule teoriche su un filtro crossover reale. \\ \\
Il filtro crossover è un circuito RLC costituito da due rami: uno con un filtro passa basso realizzato mediante un
induttore (ramo \textbf{Woofer}), l'altro con un filtro passa alto costituito da un condensatore (ramo
\textbf{Tweeter}).

\begin{figure}[H]
  \centering
  \includegraphics[width=8cm]{Schema_crossover.png}

  \caption{Schema del circuito realizzato}
  \label{fig:schema_circuito}

\end{figure}

Le due resistenze (tal volta dette \textit{resistenze di carico}) nella Fig.\,\ref{fig:schema_circuito} simulano degli
altoparlanti (da qui i nomi dei rami), e sono da considerarsi uguali a meno delle incertezze.

Le componenti di un segnale oscillante in ingresso, vengono scalate in ampiezza sui rami del circuito in base alla loro
frequenza.
Il ramo del \textit{woofer} presenta un'attenuazione progressiva delle componenti ad alta frequenza, mentre il ramo
del \textit{tweeter} di quelle a bassa.\\
La frequenza alla quale un segnale è ripartito in egual modo sui due rami è detta \textit{frequenza di cross} e dipende
dai valori della capacità e dell'induttanza del circuito.
Per calcolarla basta eguagliare le impedenze dei due rami, e si ottiene facilmente:

\begin{equation}
  \label{eq:f_cross}
  f_{cross} = \frac{1}{2 \pi \sqrt{LC} }
\end{equation}

dove $L$ è l'induttanza della bobina sul ramo Woofer e $C$ la capacità del condensatore sul ramo Tweeter.

Sempre alla frequenza di cross, si ha anche uno sfasamento dei due rami rispetto al generatore uguale e opposto: il
Woofer anticipa, il Tweeter ritarda.

\begin{align}
  \phi_{woofer} &= \arctan(-\frac{\omega L}{R}) \label{eq:p_woofer} \\
  \phi_{tweeter} &= \arctan(\frac{1}{\omega RC}) \label{eq:p_tweeter}
\end{align}

Dove $L$ e $C$ sono definiti come nell'Eq.\,\eqref{eq:f_cross}, $\omega$ è la pulsazione dell'onda in ingresso e $R$ il
valore delle resistenze sui due rami.
Inoltre si ha la seguente espressione per la differenza di fase:

\begin{equation}
  \label{eq:p_diff}
  \phi_{diff} \; = \; \arctan(\frac{\omega L}{R}) + \arctan(\frac{1}{\omega RC})
\end{equation}

Alla frequenza di cross questa vale:

\begin{equation}
  \label{eq:p_diff_cross}
  \phi_{cross} \; = \; 2 \arctan(\frac{1}{R} \sqrt{\frac{L}{C}})
\end{equation}


\end{document}