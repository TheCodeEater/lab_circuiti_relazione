% Preamble
\documentclass[../Relazione_circuiti]{subfiles}

% Packages

\graphicspath{{\subfix{../images/}}}

% Document
\begin{document}

Usando il formalismo fasoriale e la legge delle maglie sui due rami si ha:

\begin{equation*}
  \overrightarrow{I_{w}} = \frac{\overrightarrow{V_{in}}}{R+i \omega L} \qquad \quad %
  \overrightarrow{I_{t}} = \frac{\overrightarrow{V_{in}}}{R+\frac{1}{i \omega C}} \qquad \quad %
    \overrightarrow{V_{w}} = \frac{R \overrightarrow{V_{in}}}{R+i \omega L} \qquad \quad %
  \overrightarrow{V_{t}} = \frac{R \overrightarrow{V_{in}}}{R+\frac{1}{i \omega C}}  
\end{equation*}

dove \textomega \,\ è la pulsazione, L induttanza, C capacità, R resistenza, $\overrightarrow{V_{in}}$ la tensione in ingresso, $\overrightarrow{V_{w}}$ tensione sulla resistenza di carico del woofer, $\overrightarrow{V_{t}}$ sul tweeter e \textit{i} l'unità immaginaria.
Calcolando modulo delle tensioni e dividendo per R, si ottengono le Eq.\, \eqref{eq: gains}. Calcolando la fase si ottiene Eq.\, \eqref{eq: p_w_t}.

\end{document}